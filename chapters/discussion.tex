\chapter{Discussion and Open Questions}

\section{Traffic cover}
\label{sec:traffic_cover}
In the case study, we covered the traffic of push notifications by padding dummy packets and sending packets constantly (every 5 second). While constant padding is  a safe countermeasure against traffic analysis, the scheme may consume much more power than necessary since the interval of people visiting and thus non-cover message is usually far larger than 5 seconds. 

Prior research has proposed efficient ways of padding traffic. Perhaps most notably, Juarez \textit{et al.} \cite{juarez2016toward} proposed an adaptive padding algorithm and proved its effectiveness against traffic analysis. We consider it possible to adapt a similar algorithm in our system as a possible future enhancement.

\section{Decentralized push notification}
\label{sec:decentralized_push}
Ideally, our system would like to avoid all points of centralization and push notification is especially challenging since such alerting service are highly centralized. It is nearly impossible to push a notification with out using \textit{APNS (Apple Push Notification Service)} on an iOS device. It is somewhat easier on Android devices as Google permits the use of third-party push notification services. Unfortunately, most Android push notification services are designed to be 1-to-many systems (that is, one app developer pushes to many user devices), which makes it a bit difficult for private IoT devices to utilize such service.

In our case study, we used Firebase Cloud Messaging (formerly known as Google Cloud Messaging) as the push notification service provider. FCM is also designed as 1-to-many, and requires credentials for message senders. Therefore, in order to get a private channel for their own message, the users will have to register their own API key, include the key and compile the APK themselves. We consider the inconvenience a shortcoming of our system.

It is worth mentioning that implementing a private channel for push notifications is possible. A naive solution would be keeping a TCP socket open over Tor to receive notifications. Kollmann \textit{et al.} \cite{kollmann2017cost} discussed about the possibility and cost of push notifications using Tor, and pointed out that such solution is feasible but would consume more resource than most commercial push notification services.

For a proof-of-concept, we also implemented an experimental option for users to adapt such way of push notifications. In particular, we constructed our own push notification service for Android in which all communication occurs over the Tor network. Although we still reply on centralization (i.e., the push notification server), the use of Tor obfuscates clients' (the doorbells' and Android devices') locations, and prevents the centralized service from associating particular smartphones with doorbells. Further, it prevents the centralized server from easily enumerating the number of devices. We believe that such a privacy-preserving push notification is valuable for a number of uses outside of our IoT doorbell proof-of-concept. Experiments show that keeping the connection active for 4 hours consumes about 2\% of the device's battery. While the option is not ready for practical use (as there have been some unimplemented system-level issues, e.g. handling messages when the device is unavailable or the app has exited), it shows the possibility to deploy a fully decentralized push notification service with our system. We consider the full implementation to be one of the future enhancements.

\section{Storage}
In our proof-of-concept implementation, we did not add in the capability to store user videos. The videos are immediately discarded after being served. We consider it unsafe to store all the videos locally, as an adversary could physically steal or destroy the device. However, video doorbell users usually want to see what happened a few minutes or even a few days ago.

Most commercial video doorbells have applied the solution that they stored the video in their cloud server (and charge the users service fees). In this setting, the user's privacy completely depends on how the provider handles their information.

Another solution would be taking advantage of distributed storage. It is possible for users to form a distributed storage network, taking advantage of Distributed Hash Table \cite{stoica2001chord}. Thanks to DHT's property of being fault tolerant, it makes saving user data among all system users possible. There have been several approaches to apply DHT in IoT devices \cite{fabian2014privacy} \cite{paganelli2012dht}.

A practical problem with our system adapting distributed storage would be the size of data. Unlike most other IoT devices, video doorbells need to store and transmit large chunks of data (i.e. videos), which makes it difficult to implement in terms of fault tolerance.


\section{IoT devices in untrusted network}
In this thesis, we consider the home IoT device to be located in a trusted network (i.e. the home network). However, there are scenarios in which such devices will also be located in untrusted networks, for example the user is carrying the device while traveling. In such cases, additional attack surfaces and vulnerabilities are introduced as attackers can easily identify the IoT devices, analyze and manipulate the traffic. While such scenarios are not studied in this thesis, we consider it possible to adapt our solution with some additional security measurements. 

\section{IoT-to-IoT communication}
There is another communication model where IoT devices communicate with each other. This usually happens when a user has multiple home IoT devices and enables automatic workflows. While such a communication model is not covered in this thesis, it is possible to extend our solution so that each IoT device hosts its own onion service.