\chapter{Introduction}

\section{Background on IoT devices}

\section{Motivation: need for privacy-preserving framework}

\subsection{Privacy threats of IoT devices}

\subsection{Challenges}

\section{Overview of approach}

\section{Scientific questions and contributions}

\section{Background on Tor}
\label{sec:torbackground}
The Tor (The Onion Router) network, which was developed in the 1990s and deployed in 2002, is an overlay protocol to route traffic through multiple servers and encrypt it each step of the way (\cite{torproject}, \cite{chaabane2010digging}). By directing internet traffic through the onion network which consists of several thousands volunteer overlays, Tor conceals users' location and usage from adversaries.

Tor allows the consumption of \textit{Hidden Services}. Such services are only reachable within the onion network by users running the Tor client. Hidden services are identified by \textit{.onion} addresses.

The security features of Tor makes it an attractive solution for IoT communications. Besides, Tor has a few great usability features including NAT piercing and DoS prevention, which are very necessary for IoT devices. Furthermore, the system of Tor is actively maintained and researched. Thus, it is likely that flaws and vulnerabilities in the system can be fixed in reasonable time.