\chapter{Related work}

In this chapter we present prior research related to IoT security.  We present the previous approaches and discuss the difference between our approach and existing work.

In addition to security, we also review previous research related to IoT \textit{anonymity}. There have been several papers that attempt to combine the Tor network with IoT systems to achieve anonymity, but the problem has not been studied in great depth, as we discuss below.

\section{IoT security}

Privacy-preserving architecture for IoT devices has been studied widely. Song \textit{et al.} \cite{song2017privacy} design a privacy-preserving communication protocol for IoT applications. Their protocol enables IoT appliances and sensors to securely communicate with a central controller and ensures data integrity and authentication by incorporating MAC (Message Authentication Code) to the data transmission. In contrast, our approach uses onion network to achieve secure communication.

Fabian \textit{et al.} \cite{fabian2014privacy} introduce a privacy-preserving P2P data infrastructure for IoT devices based on the Octopus distributed hash table \cite{wang2012octopus} and measured the efficiency of their infrastructure (latency) using simulated network. While our design does not cover storage issues (the data are only stored locally and stashed immediately after served to the user), it is possible to extend our design by adapting their solution.

Apthorpe \textit{et al.} \cite{apthorpe2017smart} review privacy vulnerabilities in encrypted IoT traffic of four commercial IoT services. In addition, they develop a strategy to infer consumer behavior from rates of IoT traffic, which enables a passive network observer to retrieve information even when the traffic is encrypted. In this thesis, we use two strategies to avoid such traffic analysis. First, when possible, we use Tor to obfuscate the network location and traffic of IoT devices. Second, we apply cover traffic to make traffic analysis more difficult.

Alrawi \textit{et al.} \cite{alrawi2019sok} give an overview of security study challenges in modern IoT systems. In their work, they propose a modeling methodology to study home-based IoT devices and evaluate their security posture based on component analysis. In this thesis, we propose a privacy-preserving framework for home IoT devices by using the Tor and other technique to achieve privacy and decentralization.



\section{Anonymity and IoT}

Moving towards anonymous communication for IoT systems, there have been approaches utilizing onion networks. Hiller \textit{et al.} \cite{hiller2019tailoring} propose a mechanism to tailor onion routing to IoT by bridging the protocol incompatibilities and securely offloading expensive computations to an external server owned by the IoT device owner. Their work focuses on solving problems of incompatible protocols and constrained resources. While our approach propose a framework for privacy-preserving IoT functionality, Hiller \textit{et al.}'s optimization is also applicable to our approach to provide a more generalized design for privacy-preserving IoT systems.

Hoang \textit{et al.} \cite{hoang2015tor} discusses about the challenges and benefits of using Tor network to secure smart home appliances. They list a number of vulnerabilities that IoT users are facing, and show how Tor-based communication can help users protect their privacy. In this thesis, we present a concrete design and implementation for a specific type of IoT device (an Internet-enabled video doorbell).

Focusing about enhanced security communication for IoT addressing and connectivity, Baumann \textit{et al.} \cite{baumann2018utilising} discuss how utilizing Tor benefits in environments behind firewalls, proxies and NAT. They propose a prototype implementation of am Internet-enabled 3D printer using a RaspberryPi as the hardware. Our approach provides an implementation for a different type of IoT device. In addition, we describe a framework for enabling security and privacy features for IoT devices.
