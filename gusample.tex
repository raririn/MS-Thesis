% File 'gusample.tex' -- Mark Maloof, Michael A. Covington, Isidor Ruderfer
% (Revised 15 December 2010)
% This is an example of how to format a thesis or dissertation using LaTeX 2e
% with 'guthesis.sty' VERSION 3.4 OR LATER.

\documentclass[12pt]{report}

\usepackage{url}
\usepackage[numbers,sort]{natbib}
\usepackage[dvips]{graphicx}
\usepackage{amsmath}
\usepackage{amssymb}
\usepackage{algorithm}
\usepackage{algorithmic}

% If there are any other \usepackage commands, put them here.

\usepackage{guthesis} % Must be the last package

\newtheorem{theorem}{Theorem}[section]
\newenvironment{proof}[0]{\textit{Proof.}}{}
\newcommand{\qed}{\hfill $\Box$}

% To comment out multiple lines of text.
\long\def\comment#1{}

\title{Example of a Thesis \\
       Formatted with LaTeX \\
       Using the Georgetown University \\
       Style Macro Package, Version \guthesisversion}

\author{Marcus A. Maloof}

\previousdegree{B.S.}

\thisdegree{Master of Science}  % or Doctor of Philosophy, etc.

\thisdiscipline{Computer Science}

\thesistype{Thesis}     % or Dissertation

% defense or approval date, not today's date...
\thesisday{15}
\thesismonth{December}
\thesisyear{2010}

\professor{Abraham Baldwin}
%\secondprofessor{Alonzo Church}   % Only if you have 2 major professors!

\fulltitle{Example of a Thesis Formatted with LaTeX Using the Georgetown
           University Style Macro Package, Version \gustyversion}

\indexwords{Word~processing, Computer~typesetting, Computer~graphics,
            Style~sheets, Typography, Dissertations, Theses~(academic)}

\dean{Timothy A.\ Barbari}

\memberi{Benjamin Franklin}
\memberii{Isaac Newton}
% Use \memberiii, \memberiv, \memberv for up to 3 more members if needed.

\begin{document}

\pagenumbering{roman}

\maketitle    % Creates title page, copyright page if any, and approval page.

\begin{abstract}
This is the abstract, a brief summary of the contents of the entire thesis.
It is limited to 350 words.

The abstract page(s) are not numbered and are not necessarily included
in the bound copies.  Likewise, the signature page is not counted in
page numbering because not all copies contain it.

Throughout this sample thesis, {\bf please note
that the layout obtained with \LaTeX\ is not meant to be a
perfect duplicate of the Microsoft Word examples in the \emph{Graduate
School Style Manual.}}  \LaTeX\ has additional typographic tools at its
disposal, such as {\sc small capitals} and various subtle adjustments
of spacing, which are used by the Georgetown \LaTeX\ style sheet in
accordance with the standard practices of the book-printing industry.

The index words at the bottom of the abstract should be chosen carefully,
preferably with the help of one or two of your colleagues.
They are the words by which people will find your thesis when searching
the scientific literature.
If you want to get credit for your ideas, be sure to choose a good set of
index words so that people doing related work will know about yours.
\end{abstract}


\chapter*{Dedication}

The Dedication is optional, but if it is included, it should have
a roman numeral page number but not be included in the table of
contents.  To achieve that, we declare it as a \verb"\chapter*" in \LaTeX.


\pseudochapter{Acknowledgments}

In a real thesis, this section would contain acknowledgments such
as, ``This work was funded by National Science Foundation Grant
Number AAA-00-00000 (Benjamin Franklin, Principal Investigator),''
and ``I would like to thank John Doe for helping me proofread my
thesis and Mary Roe for drawing my graphs.''

The acknowledgments are included in the table of contents but do
not have a chapter number.  To achieve that, we
declare them to be a \verb"\pseudochapter" (which is defined only in
\verb"guthesis.sty").

I (Mark Maloof) would like to thank Michael Covington for developing
this \LaTeX\ style sheet for the University of Georgia.
Zachary Hunter deserves special thanks for revising the UGA style
file to work with the new version of \LaTeX\ (2e) as well as the
previous version (2.09).
The revisions in
Version 3.0 of \verb"guthesis.sty" are
largely the work of Isidor Ruderfer.
Other credits appear in the style sheet itself.

\pseudochapter{Preface}

If your thesis has a preface, this is where it goes.
A preface is not an introduction, and most theses do not need them.


\tableofcontents

\listoffigures  % Optional - Omit this line if you don't want a list of figures.
\listoftables   % Optional - Omit this line if you don't want a list of tables.

\newpage

\pagenumbering{arabic}  % Ordinary pages have Arabic numerals.

\chapter{What \LaTeX\ is all about}

This document is an example of how to format a thesis or dissertation
using \LaTeX\  and get results acceptable at Georgetown University.

\LaTeX\ (with its parent \TeX)
has two major advantages for academic use.  First, to a remarkable
degree it makes design decisions automatically.  The author supplies
only the words of a text, and \LaTeX\ places them on the page in an
aesthetic manner, avoiding rivers and awkward breaks.  In this respect
LaTeX is like a very intelligent typist or typesetter.

Second, \LaTeX\ can typeset complex mathematical formulas such as
\[
\sum_{i=1}^{\infty} x^{y+z} = \frac{p+q+r}{s+t+u+v}
\]
both displayed (as shown above) and in the text, as in
$\sum_{i=1}^{\infty} x^{y+z} = \frac{p+q+r}{s+t+u+v}$.
This makes \TeX\ and \LaTeX\ indispensable for mathematicians, physicists,
and the like.

LaTeX also has built--in formats for other kinds of displayed material
such as verse,
\begin{verse}
Freude, sch\"{o}ne G\"{o}tterfunken  \\
Tochter aus Elysium,                 \\
Wir betreten, feuertrunken,          \\
Himmlische, dein Heiligum!          \\
Deine Zauber binden wieder            \\
Was der Mode streng geteilt\dots
\end{verse}
and direct quotations:
\begin{quote}
The society that scorns excellence in plumbing, because plumbing is a
humble activity, and tolerates shoddiness in philosophy, because
philosophy is an exalted activity, will have neither good plumbing nor
good philosophy.  Neither its pipes nor its theories will hold water.\\
\hspace*{\fill} --- John Gardner, {\em Excellence}
\end{quote}
If you wish, quotes and other displayed material can be single--spaced;
here is an example of how that is achieved:
\begin{verse}
\begin{singlespace}
Yes, I wrote ``The Purple Cow.'' \\
I'm sorry now I wrote it. \\
But I can tell you anyhow \\
I'll kill you if you quote it! \\
\hfill --- Anonymous?
\end{singlespace}
\end{verse}
Well, maybe it's not as anonymous as it looks.

Computer scientists use LaTeX's ``verbatim'' format to display portions
of computer programs in text, like this:
\begin{verbatim}
1   GO TO 2
2   GO TO 1
3   PRINT "THIS STATEMENT WILL NEVER EXECUTE"
4   END
\end{verbatim}
There you have it.%
\footnote{This is a footnote. Notice that footnotes are single--spaced
even though the text is double--spaced. Long footnotes are discouraged;
either make important points in the text or leave them out.}

Whenever you quote parts of a computer program in English text, they
should be set off by using typewriter type.  The \verb"PRINT" statement
in the program above will never execute because the \verb"GO TO"
statement above it keeps execution from reaching it.



\chapter{Five Important Notes}

\section{\LaTeX\ is not a typewriter --- nor an ordinary word processor}

\emph{The format that you get with \LaTeX\ is more elegant
than the simpler format in the Graduate School style manual and is not
intended to match it perfectly.}

\LaTeX\ makes available to you \emph{all} of the tools of fine book printing.
For example, \LaTeX\ has more degrees of
vertical spacing available than other word processors do, and it uses them
effectively.  Likewise, \LaTeX\ provides {\sc Small Capitals} for situations
where other software would only provide ALL CAPITALS.

The format produced by \verb"guthesis.sty" has been reviewed by the Graduate School.
If your thesis fails the format check, make sure you have followed the
instructions for \verb"guthesis.sty" (contained in the file \verb"gusample.tex")
and the instructions for \LaTeX.

\section{You can't do without the book!}

\emph{You are not expected to be able to use \LaTeX\ without a manual.}
Don't even try --- you'll waste a lot of time.
The definitive manual for \LaTeX\ is:
\begin{quote}
Lamport, Leslie (1994) \emph{\LaTeX: A Document Preparation System.}
2nd ed.
Reading, MA: Addison-Wesley.
\end{quote}
Alternatively, there is a good manual available on line whose author is
named Oetiker.
There are several versions.  Use a search engine to search for
``LaTeX Oetiker'' and you'll find it.

\section{The {\tt guthesis.sty} style sheet adds to \LaTeX\ but does not replace it}

``How do I do so-and-so in \verb"guthesis.sty"?''  Almost always the same way as
in plain \LaTeX.
The Georgetown University style sheet, called \verb"guthesis.sty", adds features
to \LaTeX; it does not take away what is already there.


\section{Ragged right margins are optional}

{\raggedright
\parindent 0.25in
This section is typeset in ``ragged right'' format, in which \LaTeX\ has
been told not to force a straight right margin.

``Ragged right'' is preferable if your thesis contains a large number of
words likely to be hyphenated incorrectly, formulas, web addresses, or
the like.  To get it, put the two lines
\begin{verbatim}
\raggedright
\parindent 0.25pt
\end{verbatim}
anywhere after the {\tt documentclass} command and before
\verb"\begin{document}".

You can also reduce the amount of hyphenation, without eliminating it
totally, by following the instructions in the \LaTeX\ manual.
} % raggedright

\section{Please let \LaTeX\ do the layout...}

Please do not use \verb"\vspace", \verb"\hspace", and other \TeX\ commands
that specify the precise location or spacing of items on the page.
Leave the layout to \LaTeX\ and \verb"guthesis.sty".
Otherwise, very ugly things can happen.

Your thesis is like a volume in a series.
It should look like the other theses and dissertations that were typeset
using the same technology.
Just as different volumes of the \emph{Encyclopaedia Britannica} have exactly
the same margins and typography, so should different theses from the same
university, as far as technology permits.


\section{Please use floating figures and tables...}

Please use the \verb"figure" and \verb"table" environments for your figures
and tables.
A picture or table should not interrupt the text the way a mathematical
formula does.
Instead, it should ``float'' to the nearest convenient location---which
\LaTeX\ will take care of automatically---and should be identified by
number.



\chapter{Frequently Asked Questions About Typing Georgetown Theses with
\LaTeX\ and {\tt guthesis.sty}}

\section{\LaTeX\ itself}

\subsection{What is \LaTeX?}

\LaTeX, by Leslie Lamport, is an extension of \TeX, the computer typesetting
system designed by Donald Knuth.  This software system is used to typeset
books, journals, papers, and theses in the mathematical sciences.
Knuth and Lamport had two goals in designing it:
\begin{itemize}
\item To use the computer to equal the quality of the best conventional
typesetting.
No longer does ``word processing'' mean ``a poor substitute for real
printing.''
\item To separate the jobs of the \emph{author} and the \emph{typesetter.}
\end{itemize}
Other word processors turn the screen into a blank piece of paper and let
you type on it.  \LaTeX\ is not your \emph{typewriter,} it's your
\emph{typist.}  For example, to mark the beginning of a chapter, you
type something like this:
\begin{verbatim}
\chapter{Literature Review}
\end{verbatim}
and \LaTeX\ automatically determines what a chapter heading should look like.
It does this by consulting a \emph{style sheet.}  You do not have to worry
about whether you've hit Enter the same number of times at the beginning of
each chapter, or whether all the chapter headings are centered and the
margins are correct.  That's the (automated) typist's job.

\subsection{How is ``\LaTeX'' pronounced?}

The X in \TeX\ is actually a Greek chi, with the sound of \emph{ch} in
Scottish \emph{loch} or German \emph{ach}.  Thus, \TeX\ is pronounced
``tekh.''  It is short for Greek \emph{tekhn\={e}} ``art, craft.''

\LaTeX\ is pronounced \emph``lah-tekh'' in the Northeast, \emph``lay-tekh''
in England and the South, and several different ways in the West.

\subsection{What's it like to run \LaTeX?}

\LaTeX\ is \emph{not} a single integrated piece of software.  It has several
parts.  The normal process for typing and printing a paper is as follows.
On finding an error or discovering that a change is needed, you can go back
to any earlier step at any time.
\begin{enumerate}
\item Use a text editor (Windows Notepad or whatever text editor you like)
to type your document on a file whose name ends in {\.tex}.
\item Go to a command prompt and run {\tt latex} to create a {\tt .dvi} file.
\item Optionally, preview the {\tt .dvi} file on-screen.  Some of us skip
this step.
\item Run {\tt dvipdf} to convert the {\tt .dvi} file to PDF.
\item Use Adobe Reader to view the PDF file and print it, if desired.
\end{enumerate}

\subsection{Who benefits the most from using \LaTeX?}

Those who need to typeset mathematical formulas or computer programs;
those whose theses are likely to be published
by a book publisher; and those who plan to submit their theses electronically.

\subsection{Who should \emph{not} use \LaTeX?}

If you consider yourself a ``computer dummy'' and can use only the simplest
software, or if you do not have the time or patience to learn a new set of
technical skills, you won't like \LaTeX.  On the other hand, you probably
will not type your thesis correctly with a conventional word processor
either.  (Let's face it, typing a thesis correctly is a technical challenge,
no matter what software you do it with!)  You should hire a typist.

\section{\LaTeX\ versions}

\subsection{What are Mik\TeX, em\TeX, te\TeX, etc.?}

These are implementations of \LaTeX\ for particular computers, incorporating
Knuth and Lamport's original computer programs plus various tools to make
them easier to use or more versatile.
All of them are 100\% compatible with the original \LaTeX.

\subsection{Is there \LaTeX\ for Macintosh, Linux, Amiga, Sun, BeBox...?}

Yes.  \LaTeX\ has been ported to a huge variety of computers and produces
\emph{identical} output on all of them, using \emph{identical} file formats.
If you have Linux, you almost certainly already have \LaTeX.

\subsection{What is \LaTeXe?}

The version of \LaTeX\ that has been in use for the last several years.
It was preceded by \LaTeX\ 2.09.  Those are the only two version of
\LaTeX\ that have been distributed widely.  There will eventually be
a \LaTeX\ 3, but not very soon.

\subsection{Where do I get \LaTeX?}

You can download it free of charge from various sites;
Be careful about Googling \LaTeX.
It's best to go to the \TeX\ Users Group web site:
\url{http://www.tug.org}.

\subsection{Where is the documentation for \LaTeX?}

In the book \emph{LaTeX: A Document Preparation System,} by Leslie Lamport,
2nd edition, published by Addison-Wesley.  \textbf{You must buy this book.}
Other books about \LaTeX\ are useful but are not the official guide.

\subsection{Where can I get help with \LaTeX?}

First, read Lamport's book.  \emph{You cannot get along without it.}

Second, you can ask questions on the newsgroups {\tt comp.text.tex}
(international).

Third, if you have problems that are specific to {\tt guthesis.sty},
and particularly if {\tt guthesis.sty} does not appear to be meeting
Georgetown's thesis format requirements, please contact me
(Mark Maloof, maloof@cs.georgetown.edu).


\section{The Georgetown University style sheet}

\subsection{What is {\tt guthesis.sty}?}

A style sheet for \LaTeX\ that makes it follow the format for
Georgetown University theses and dissertations.

\subsection{Where do I get {\tt guthesis.sty} and how do I install it?}

Download it from \texttt{http://www.cs.georgetown.edu/academics/guthesis.zip}.

Install it by putting it in the same directory as your thesis, or in
directory {\tt texmf/tex/latex} of your \TeX\ system.

\subsection{How do I use {\tt guthesis.sty}?}

For a quick summary, see Fig.\ \ref{gutemplate}.
For details, see the file {\tt gusample.tex}.
For a skeleton document, see the file {\tt template.tex}.

\begin{figure}
\begin{center}
\begin{minipage}{5in}
\tiny
\begin{verbatim}
     \documentclass[12pt]{report}
     \usepackage{guthesis}

     % The following 2 lines are OPTIONAL to turn off all hypenation
     % and print with "ragged" right margins
     \raggedright
     \parindent 0.25in

     \title{Title of Thesis \\
            Broken into Lines \\
            As Appropriate}

     \fulltitle{Title of Thesis Not Broken Into Lines}

     \author{Your Name Here}

     \previousdegree{B.A., Name of University, 1977 \\
                     M.S., Some Other University, 1978}

     \thisdegree{Master of Science} % or Doctor of Philosophy, etc.
     \thesistype{Thesis}        % or Dissertation
     \thesisday{5}              % put the day of defense or approval here...
     \thesismonth{May}          % put the month of defense or approval here...
     \thesisyear{2005}          % put the year of defense or approval here...

     \indexwords{keywords for indexing your thesis in library catalogs}

     \professor{Abraham Baldwin}     % Major Professor
     \secondprofessor{Alonzo Church} % Co-Major Professor if you have one
     \memberi{Benjamin Franklin}     % Committee member
     \memberii{Isaac Newton}         % Committee member
     % Use \memberiii, \memberiv, \memberv for up to 3 more members if needed.

     \dean{Timothy A.\ Barbari}       % Dean of the Graduate School

     \begin{document}
     \pagenumbering{roman}   % Use Roman numerals until chapters start.

     \begin{abstract}
     Text of abstract goes here.
     \end{abstract}

     \maketitle    % Creates title page, copyright page if any, and approval page.

     \chapter*{Dedication}
     Dedication, if any, goes here.
     It is a "\chapter*" so it is not in the table of contents.

     \pseudochapter{Acknowledgments}
     Text of acknowledgments goes here.
     It is a "\pseudochapter" so it goes in the table of contents but has no chapter number.

     \tableofcontents
     \listoffigures   % Optional - Omit this line if you don't want a list of figures.
     \listoftables    % Optional - Omit this line if you don't want a list of tables.

     \newpage
     \pagenumbering{arabic}  % Ordinary pages have Arabic numerals.

     \chapter{Title of Chapter Goes Here}
     Text of first chapter goes here.

     \chapter{Title of Chapter Goes Here}
     Text of second chapter goes here.

     \appendices  % Indicates that appendices follow.  If there's only one, use \appendix instead.

     \chapter{Title of First Appendix}
     Text of first appendix goes here.

     \chapter{Title of Second Appendix}
     Text of second appendix goes here.

     \bibliographystyle{plainnat}
     \bibliography{thesis}  % Pulls entries from thesis.bib

     \end{document}
\end{verbatim}
\end{minipage}
\end{center}
\caption{Template for formatting a thesis with {\tt guthesis.sty}.}
\label{gutemplate}
\end{figure}


\subsection{Do I have to get {\tt guthesis.sty} before typing my thesis?}

No.  You can type your thesis using the \LaTeX\ {\tt report} documentclass
(described in Lamport's book); this will enable you to produce neat,
single-spaced copies of the work in progress.  When it's finished, get
{\tt guthesis.sty} and make the small additions to your thesis that are described
there.


\subsection{What are the required parts of a thesis typed with {\tt guthesis.sty}?}

They are described in {\tt guthesis.sty} itself, which you can read with your
text editor.
Alternatively, you can use {\tt gusample.tex} as sample to imitate.


\section{Troubleshooting}

\subsection{Why is the text too high or low on the page?}

See Appendix A, p.\ \pageref{mgins}.


\subsection{Why is there a space before a footnote number?}

If your footnote numbers come out positioned like this $^8$ rather than
this$^8$, it's because you're leaving a space before the
\verb.\footnote. command.
\begin{verbatim}
Type like this.\footnote{Here's the note.}
Not like this. \footnote{Here's the note.}
\end{verbatim}
A useful trick is to put \verb"%" at the end of one line (to comment out
all the remaining blank space) and then \verb"\footnote" at the beginning
of the next line.

\subsection{Why is my table of contents blank or incorrect?}

You must run \LaTeX\ twice in order to get a correct table of contents.
The first time, it keeps records of where things are; the second time,
it actually generates the table of contents.  If the table of contents
is long, you may need to run \LaTeX\ three times to ensure that adequate
space is left for it.

\subsection{Why is there too much space after some of the periods?}

\LaTeX\ normally assumes that every period marks the end of a sentence,
so it leaves extra space after it.  You should normally use a required
space (\verb.~.) after every period that does not mark the end of a sentence.
Type ``\verb"T.~S.~Eliot"'' to print ``T.~S.~Eliot'' or the like.

Beginning with {\tt guthesis.sty} version 2.0 (December 2001), this is no longer
a problem because {\tt guthesis.sty} tells \LaTeX\ not to treat periods specially;
the same amount of space follows each word whether or not it ends with a
period.  This is known as ``French spacing'' and is an accepted practice
in the printing industry.

\subsection{Why do some words hang out past the right margin?}

When \LaTeX\ cannot break a line satisfactorily, it leaves a word sticking
out into the margin and gives you an ``Overfull hbox'' error message.
It is up to you to rearrange the text so that it fits.

\subsection{What is an overfull hbox?}

See previous question.


\subsection{What is a ``Package hyperref warning''?}

Most people now use the \verb"hyperref" package to generate hyperlinks to
their chapters and sections in the eventual PDF output.

This is good, but \verb"hyperref" complains violently if there is anything
other than plain text in a section title.
It also complains if you use \verb"\pseudochapter".
Just ignore the warnings.


\section{Character set and typography}

\subsection{How do I type a percent sign?}

See Lamport's book.  A quick answer:  Type ``\verb.\%.''.

\subsection{How do I type a tilde ($\sim$)?}

This character often occurs in web addresses.
In \LaTeX, when you type ``\verb.~.'' you get a blank.
(Specifically, you get a ``required space,'' a space that cannot be
broken across a line break.)
To get ``$\sim$'' type ``\verb.$\sim$.''.
Also see the {\tt url} package, which lets you write
\verb|\url{http://www.cs.georgetown.edu/~maloof}|, typeset as
\url{http://www.cs.georgetown.edu/~maloof}.


\subsection{What are \textit{italics} used for?}

All of the things that would be underlined in a handwritten document,
including titles of books, foreign words, and the like.

In linguistics, it is normal to put foreign words in italics and their
definitions in single quotes.  For example, Agatha Christie's famous
detective is named after the French word \emph{poireau} `leek'.


\subsection{What is \underline{underlining} used for?}

Almost nothing.  Roman type is not normally underlined; use italics instead.


\subsection{What is \mbox{\tt typewriter} \mbox{\tt type} used for?}

Computer program languages, whether displayed or quoted in text.  For
example, here is part of a program written in C:
\begin{verbatim}
for ( i = 100; i > 0; i-- ) {
   printf("%d bottles of beer on the wall...\n", i);
}
\end{verbatim}
It demonstrates how to use the \verb"for" statement to count down from
100 to 1.

\subsection{What is the \mbox{\tt verbatim} environment used for?}

Computer programs, as just demonstrated.

\subsection{What is \mbox{\sf sans-serif type} used for?}

Almost nothing except labels within illustrations.


\section{Bibliographies}

\subsection{How do I type the bibliography?}

See Lamport's book \cite{lamport.94}
and the examples in {\tt gusample.tex}.
\LaTeX\ lets you refer to bibliography items in your text
with markers such as
\verb|\cite{lamport.94}|.
then have \LaTeX\ automatically turn these into bracketed numbers in
the bibliography.

With {\tt guthesis.sty}, simply use the \verb.thebibliography. environment
exactly as Lamport describes it.  This will produce a bibliography in
the form of an unnumbered chapter at the end of your thesis.

\subsection{What if each chapter has its own bibliography?}

If you have bibliographies at the ends of the individual chapters, use the
environment {\tt chapterbibliography} instead of {\tt thebibliography}.
It works \emph{exactly} the same way except that the bibliography
becomes a normally numbered section, not an unnumbered chapter.
There is an example of this in {\tt gusample.tex}.

The {\tt chapterbibliography} environment is provided by {\tt guthesis.sty}.

\subsection{What if I don't like bracketed numbers?}

Here is an example of a trick to get \LaTeX\ to print a bibliography without
bracketed numbers.  Basically, you are telling \LaTeX\ to put the author's
name in place of the bracketed number.  Note that this involves using
\verb.\item. rather than \verb.\bibitem..
\begin{verbatim}
\begin{thebibliography}{}

\item[Covington, Michael A.]
\emph{Natural Language Processing for Prolog Programmers.}
Englewood Cliffs, N.J.: Prentice Hall, 1994.

\item[O'Keefe, Richard A.]
\emph{The Craft of Prolog.}
Cambridge, MA: MIT Press, 1991.

\end{thebibliography}
\end{verbatim}
When doing this,
don't use the \verb"\cite" command; instead, handle your references manually.
You can do \emph{exactly} the same thing with \texttt{chapterbibliography}.


\section{Layout and illustrations}


\subsection{How do I turn off justification?}

Justification means printing with a straight right margins.  It is not required
for Georgetown University theses, and if your text contains many formulas, web
addresses, or other unbreakable items, you may get considerably neater results
by turning it off.  To do this, issue the command
\begin{verbatim}
\raggedright
\end{verbatim}
immediately after \verb.\begin{document}..

You may need to turn off justification only within a bibliography.
In that case, you should put the \verb.\raggedright. command right after
\verb.\begin{thebibliography}.
or right after
\verb.\begin{chapterbibliography}.
as the case may be.  It will then affect only the bibliography.

\subsection{How do I put a picture into my thesis?}

See Lamport's book.  Here's the basic process...
\begin{enumerate}

\item Learn the difference between a vector (``draw'') program and a bitmap
(``paint'') program.  Vector programs,
such as xfig, OpenOffice Draw, Corel Draw, and Micrografx Windows Draw,
tell the computer to draw lines at particular positions;
they are the right tool for generating diagrams of all types.  Bitmap
programs are \emph{only} for working with digitized photographs and the
like; their output has an unpleasant stairstep appearance when enlarged
or resized.

\item Produce professional-quality artwork.  (You may want to hire a
professional illustrator.)
Artwork in your thesis should look as good as the artwork in published books.

\item Save your artwork as an encapsulated PostScript (EPS) file
\emph{with no TIFF header}.
(The drawing software may ask you whether you want a TIFF header; say no.)

\item Use the \texttt{graphics} package (described in Lamport's book) or the
\texttt{epsf} package to incorporate your art into your \LaTeX\ document.
One way to do this is as follows:
\begin{enumerate}
\item Add the command \verb.\usepackage{graphicx}. right after
\verb.\documentclass..
\item Use the command \verb"\centerline{\includegraphics{xxxxxxx.eps}}"
(with the appropriate filename substituted) to put the picture in your
document.  Normally this will be within a \verb.figure. environment as
described in Lamport's book.
\end{enumerate}
\end{enumerate}

\subsection{How do I put a Windows screen shot into my thesis?}

When writing about software, you many need a picture of the computer screen
with a program running.
Under Windows 95 and up, you can ``take a picture'' of the screen, by
pressing Print Screen
(or Alt-Print Screen if you only need the current window).
This puts a copy of the screen into the Windows clipboard.  Then open up
your favorite paint program (bitmap program) and choose Paste.  Edit
the picture to your satisfaction, save it, and export it as encapsulated
PostScript.  For the rest of the process, see the previous question.


\chapter{The Craft of Scholarly Typing}

\section{An exhortation}

If you have come to the University for a graduate degree, scholarly writing
is one of the crafts
in which you are being trained.
This includes scholarly typing and word processing.
We assume that you will be writing about your subject for the rest of your
life.
This is more likely to be true than you realize.

\LaTeX\ provides you with all the tools of fine book printing.
This includes quite a few things that were not available on typewriters.
Accordingly, when using \LaTeX, you must learn some new things about printed
language.


\section{Some common punctuation errors}

\subsection{Hyphens and dashes}

\LaTeX\ distinguishes four kinds of horizontal lines between characters.

\begin{itemize}
\item The minus sign, $-$ (typed \verb"$-$"),
is used only in math mode.  Always go into math mode when you want to mark
a number as negative.

\item The hyphen, - (typed \verb"-"), is used between parts of a word, as in
``so-called'' or ``heavy-handed.''

\item The short dash (en dash), -- (typed \verb"--"), goes between numbers
in sequences, as in ``1861--65.''

\item The long dash (em dash), --- (typed \verb"---"), goes between parts
of sentences and may be set off by spaces --- like this.
In good writing, it is uncommon; as alternatives, consider using a semicolon
or starting a new sentence.
\end{itemize}

\subsection{Quotation marks}

In \LaTeX\, you must type your quotation marks \verb"``like this''" in order
to get them to print correctly ``like this.''  It is not correct to type
"like this."

Periods and commas ``slide under'' quotation marks so that they adhere to
the words within them.
Correct usage looks ``like this.''  Incorrect usage looks ``like this''.

In computer science and linguistics, this rule sometimes does not apply,
because it may be important to show that the period or comma is not part
of the quoted material.  If you cannot get around this by rearranging the
sentence, it is acceptable to put the period or comma outside the quotes.

Single quotation marks (`like this') are used in linguistics to give the
meanings of words.  Thus, French \emph{chateau} `castle' comes from Latin
\emph{castellum} `fortress.'


\subsection{Punctuation around mathematical formulae}

Do not let the punctuation of your sentence intrude into a displayed
mathematical formula.
When a sentence ends with a displayed formula, end the sentence with a colon.
Thus we conclude:
\[
a = \sum_{x=1}^\infty p(x) = q(x)
\]
The formula does not end with a period.


\section{Knowing the character set}

Part of knowing your subject is knowing its notation.
If you are vague on the difference between $\emptyset$ and $\phi$, or
$\omega$ and $w$, consult some well-produced reference books
(not scruffily typed conference papers).

Remember that in good printing, \emph{underlining is almost never used.}
Underlining is the handwritten way of indicating italics.
Now that you have italics, use them.
Instead of wavy underlining (for vectors, etc.), use boldface.


\section{Figures and tables}

In a well-produced book, figures and tables do not interrupt the text like
mathematical formulae.
Instead, they ``float'' to the nearest convenient location and are identified
by number.
If there is no convenient place at the top or bottom of a page, they are
placed on a separate page.

\emph{See Lamport's book to learn how to control the placement of figures
and tables.}
Nothing is added or changed by {\tt guthesis.sty} --- \LaTeX\ still works exactly
the way Lamport says.

\begin{figure}
\begin{center}
\framebox{A picture could go here.}
\end{center}
\caption{This is an example of a figure.}
\label{myfig}
\end{figure}


\begin{figure}
\begin{center}
\framebox{A picture could go here.}
\end{center}
\caption{This is another example of a figure.}
\label{myfigtwo}
\end{figure}


Near here, you will see Figures~\ref{myfig} and~\ref{myfigtwo}.  Note that
each picture's caption comes below it.  That is achieved by putting
the \verb"\caption" command \emph{after} the figure.

\begin{table}
\caption{Some institutions and their mascots.}
\label{mytb}
\begin{center}
\begin{tabular}{|c|c|c|}
\hline
\bf Institution & \bf Location & \bf Mascot \\
\hline
University of Georgia  & Athens & Bulldawg \\
Georgetown University  & Washington  & Bulldog \\
Georgia Tech  & Atlanta  & Insect \\
UCSC  &  Santa Cruz  & Banana Slug\\
\hline
\end{tabular}
\end{center}
\end{table}

Near here, you will also see Table~\ref{mytb}.  Its caption
comes above it.  This is achieved by putting
the \verb"\caption" command \emph{before} the table.


\chapter{More Nonsense about \LaTeX}

In this chapter, we talk about more \LaTeX\ things.
Suspendisse potenti. Proin justo lorem, rutrum ac, facilisis in,
malesuada sed, ligula. Mauris lobortis lacus at nibh. Aenean vitae
odio vel odio placerat hendrerit. Suspendisse lacus lacus, tempor id,
pharetra eget, ornare sit amet, pede. Sed aliquet, justo ac elementum
pretium, arcu leo placerat est, a luctus purus diam eget arcu. Nam
augue diam, mollis a, scelerisque eget, aliquet condimentum, pede.
Vestibulum tristique lectus sed augue.

\section{Introduction}\label{sec:intro}

Aenean ut mauris luctus mauris interdum convallis. Nunc vestibulum
sodales nulla. Nulla vitae massa. Maecenas vel tellus vitae elit mattis
adipiscing. In pulvinar felis sed est. Mauris non mi. Duis ultrices dolor
ut orci. Quisque lacinia arcu et purus. Sed euismod metus nec augue.

Cum sociis natoque penatibus et magnis dis parturient montes, nascetur
ridiculus mus. Nunc dolor leo, aliquam a, placerat sit amet, accumsan
eget, dolor. Sed lacinia augue in magna. Fusce sed enim. Vestibulum et
mauris. Phasellus in lectus. Pellentesque eu elit in dolor ullamcorper
sodales. Vestibulum interdum ornare ligula. Mauris felis odio, rhoncus
sed, adipiscing fermentum, tincidunt eu, metus. Suspendisse viverra
rhoncus purus.

Fusce euismod nisi at libero malesuada consectetuer. Proin laoreet, nunc
quis hendrerit gravida, neque leo placerat sapien, et semper tortor leo
et urna. Ut tincidunt posuere tortor. Sed tristique, odio at luctus
facilisis, nulla quam rhoncus tortor, at congue lacus elit nec metus.
Sed tempor sapien ut elit. Donec ac turpis feugiat nisi porta vehicula.

\comment{
This is an example of a block comment
}

\section{Problem Statement}\label{sec:problem}

This is a test.  Blah, blah, blah. $\Pr(A) = \Pr(A) + \Pr(B|A)$.
We discuss this issue in Section~\ref{sec:method}.  Indeed, we have
\[
\sum^{n}_{k = 1}{f(k)} \; ,
\]
where $f(k)$ is my favorite function.

\section{Related Work}\label{sec:related}

Lorem ipsum dolor sit amet, consectetuer adipiscing elit \cite{breiman.ml.96}.
Phasellus non erat eu dui dignissim dictum.
Integer iaculis nulla at nisl \cite{jaynes.mebm.90}.
Proin ut enim non ipsum varius laoreet \cite{muslea.aaai.00,sahai.00}.
Integer feugiat, ante fringilla blandit
convallis, leo sapien egestas velit, non condimentum nulla sem vitae
risus. Mauris aliquam auctor quam. Sed ac enim. Donec mattis dui id
ligula. Integer vel sem eget ante cursus tristique. Nullam vel orci
vitae sem interdum placerat. In eget lectus. Donec blandit. Quisque
lacus urna, malesuada vel, mollis sit amet, rutrum nec, est. Proin
blandit ornare nibh. Duis et felis \cite{pfleger.tr.97}.

\section{Approach}\label{sec:approach}

Here is an example of code:

\begin{verbatim}
int main()
{
  cout << "testing" << endl;
  return 0;
} // main
\end{verbatim}
Text.
Plain text. {\em Emphasized text.} {\it Italicized text.}
{\bf Boldfaced text}.  {\tt typewriter text.} {\large Large text.}
Our results appear in Table~\ref{tab:results}.
In their paper \cite{muslea.aaai.00},
they discuss a lot of interesting things.

Em dash --- which is followed by an en dash -- and then we have
eloquently-hyphenated words, which are not to be confused with
the minus sign: $-$.  All are different \cite{shamos.phd.78}.

Here is a quote:
\begin{quote}
Ask not what your country can do for you.  Ask what you can do
for your country.
\end{quote}

In Figure~\ref{alg:min}, we present an algorithm for finding the
minimum component of a vector.
But there are other ways to present algorithms.
Some have started using the {\tt algorithm} and {\tt algorithmic}
packages.
See Algorithm~\ref{alg:pair} as an example.
Lorem ipsum dolor sit amet, consectetuer adipiscing elit. Nam iaculis,
felis nec semper malesuada, quam metus placerat ligula, id euismod purus
eros eget justo. Pellentesque ipsum. In hac habitasse platea dictumst.
Ut pede pede, sagittis sit amet, dapibus eu, tristique in, ipsum
\cite{kolter.icml.05}.

\begin{figure}[t]
\begin{quote}
\begin{tabbing}
XX\=XX\=XX\=XX\=XX\=XXXXXXXXXXXXXXXXXX\= \kill
 \> {\sf Algorithm}\,$( \; \vec{x} \; )$ \\[\smallskipamount]
 \>\> $\vec{x}$: vector of numbers \\
 \>\> $i$: index of minimum value \\
 \>\> $\mbox{\em min}$: minimum value \\[\smallskipamount]
\parbox{3ex}{\hfill 1.} \>\> $m \leftarrow \infty$ \\
\parbox{3ex}{\hfill 2.} \>\> {\bf for} $j \leftarrow 1, \ldots, n$
\>\>\>\> // Loop over components \\
\parbox{3ex}{\hfill 3.} \>\>\> {\bf if} $( \vec{x}_{j} < \mbox{\em min} )$ \\
\parbox{3ex}{\hfill 4.} \>\>\>\> $\mbox{\em min} \leftarrow \vec{x}_{j}$ \\
\parbox{3ex}{\hfill 5.} \>\>\>\> $i \leftarrow j$ \\
\parbox{3ex}{\hfill 6.} \>\>\> {\bf end}; \\
\parbox{3ex}{\hfill 7.} \>\> {\bf end}; \\
\parbox{3ex}{\hfill 8.} \>\> {\bf return} $i$ \\
\> {\bf end.}
\end{tabbing}
\end{quote}
\caption{Algorithm for finding the minimum component of a vector.}
\label{alg:min}
\end{figure}

\paragraph{Paragraph.} Something you may never use, kind of like
Latin.\footnote{And footnotes.}
Aliquam fermentum velit tristique turpis. Mauris mollis dapibus purus.
Quisque vulputate ullamcorper purus. Quisque aliquam rhoncus felis.
Aenean fringilla mattis mi. Ut ut massa a sem fermentum venenatis.
Fusce ut risus. Maecenas ut nisl. Praesent felis erat, tincidunt ac,
placerat vel, sodales vel, justo. Vivamus semper sollicitudin est.
Morbi id eros eu tellus dignissim viverra \cite{breiman.ml.96}.

\begin{algorithm}[tb]
  \caption{Paired Learner}
  \label{alg:pair}
  \begin{algorithmic}[1]
    \STATE {\bfseries Input:} $\{ \vec{x}_t, y_t \}^{T}_{t=1}$, $w$, $\theta$
    \vskip 0.3em
    \STATE $\{ \vec{x}_t, y_t \}^{T}_{t=1}$: training data
    \STATE $w$: window size for the reactive learner
    \STATE $\theta$: threshold for creating a new stable learner
    \vskip 0.3em
    \STATE Let $S$ be a stable learner
    \STATE Let $R_{w}$ be a $w$-reactive learner
    \STATE Let $C$ be a circular list of $w$ bits, each initially $0$
    \FOR{ $t \leftarrow 1$ {\bfseries to} $T$ }
    \STATE $\hat{y}_{S} \leftarrow S.\mbox{\sf Classify}( \vec{x}_{t} )$
    \STATE {\bfseries output} $\hat{y}_{S}$
    \STATE $\hat{y}_{R} \leftarrow R_{w}.\mbox{\sf Classify}( \vec{x}_{t} )$
    \IF{ $\hat{y}_{S} \neq y_{t} \wedge \hat{y}_{R} = y_{t}$ }
    \STATE $C.\mbox{\sf set}( t )$
    \ELSE
    \STATE $C.\mbox{\sf unset}( t )$
    \ENDIF
    \IF{ $\theta < C.\mbox{\sf proportionOfSetBits}()$ }
    \STATE $S \leftarrow \mbox{\bf new } \mbox{\sf StableLearner}()$
    \STATE $S \leftarrow R_{w}.\mbox{\sf getConceptDescription}()$
    \STATE $C.\mbox{\sf unsetAll}()$
    \ENDIF
    \STATE $S.\mbox{\sf Train}( \vec{x}_{t}, y_{t} )$
    \STATE $R_{w}.\mbox{\sf Train}( \vec{x}_{t}, y_{t} )$
    \ENDFOR
  \end{algorithmic}
\end{algorithm}

Text.  See Figure~\ref{fig:design}.
Nullam massa est, facilisis at, egestas non, lobortis at, felis.
Vestibulum eu velit ut justo commodo pretium. Quisque commodo velit
vel sem. Cum sociis natoque penatibus et magnis dis parturient montes,
nascetur ridiculus mus. Donec nec magna. Aliquam erat volutpat.

\begin{theorem}[Kolter and Maloof \cite{kolter.icml.05}]
\label{addexp.d-mistake-1}
For any time steps $\mathrm{t_1} < \mathrm{t_2}$, if we stipulate that
$\beta + 2\gamma < 1$, then the number of mistakes that \textbf{AddExp.D}
will make between time steps $\mathrm{t_1}$ and $\mathrm{t_2}$ can be
bounded by
\[
M_{t_2} - M_{t_1} \leq
\frac{\log (W_{t_1} / W_{t_2})}{\log (2/(1+\beta+2\gamma))} \; .
\]
\end{theorem}

\begin{proof}
\textbf{AddExp.D} operates by multiplying the weights of the experts that
predicted incorrectly by $\beta$.  If we assume that a mistake is made at
time step $t$ then we know that at least $1/2$ of the weight in the
ensemble predicted incorrectly.  In addition, a new expert will be added
with weight $\gamma W_t$.  So we can bound the weight with
\[
W_{t+1} \leq \frac{1}{2}W_t + \frac{\beta}{2}W_t + \gamma W_t =
\frac{1+\beta +2\gamma}{2} W_t \; .
\]
Applying this function recursively leads to:
\[
W_{t_2} \leq \frac{1+\beta+2\gamma}{2}^{M_{t_2} - M_{t_1}} W_{t_1} \; .
\]
Taking the logarithm and rearranging terms gives the desired bound, which
will h old since the requirement that $\beta + 2\gamma < 1$ ensures that
$(1 + \beta + 2\gamma)/2 < 1$. \qed
\end{proof}

\section{Experimental Study}\label{sec:study}

Always put something here, even if it just informs the reader of what
comes next.
Integer odio lorem, aliquam id, dignissim in, varius at, neque. Class
aptent taciti sociosqu ad litora torquent per conubia nostra, per
inceptos himenaeos. Maecenas vitae massa ultrices lorem aliquam pharetra.
In hac habitasse platea dictumst. Nunc consectetuer sapien. Maecenas ut
felis. Sed blandit congue mauris. Donec ornare laoreet tortor. Cum sociis
natoque penatibus et magnis dis parturient montes, nascetur ridiculus mus.
Fusce id neque ac enim lobortis aliquet. Sed a purus ac nunc feugiat
interdum. Suspendisse risus. Donec pulvinar, turpis at rutrum porta,
justo massa iaculis quam, vel pharetra leo magna et metus. Etiam tellus
dolor, volutpat et, dignissim vel, iaculis in, sem. Donec ante est,
ultricies non, eleifend in, pretium id, nisl. Fusce vitae nunc.
Vivamus egestas. Vestibulum nibh. Aliquam rutrum imperdiet diam.

\begin{figure}[t]
\centerline{\includegraphics[width=1.0in]{state.eps}}
\caption{A pretty picture.} \label{fig:design}
\end{figure}

\subsection{Method}\label{sec:method}

Lorem ipsum dolor sit amet, consectetuer adipiscing elit. Sed vitae risus.
Vivamus dapibus, quam vehicula eleifend vestibulum, pede magna iaculis
leo, quis convallis nisl mauris id arcu. Morbi tincidunt odio at libero.
Mauris molestie. Proin placerat, neque sit amet eleifend ullamcorper,
arcu sem ornare arcu, a adipiscing metus felis eu turpis. Integer bibendum
nisi placerat quam. Donec id enim. Sed neque ipsum, ullamcorper vitae,
bibendum nec, pharetra quis, quam. Curabitur suscipit aliquet ligula.
Aenean pede.

Cras molestie turpis ac lorem. Vivamus molestie porta purus. Nulla
facilisi. Lorem ipsum dolor sit amet, consectetuer adipiscing elit.
Fusce fermentum, ante in vestibulum auctor, turpis tellus aliquam est,
in accumsan massa tortor id magna. Mauris sit amet libero. Donec orci dui,
interdum in, sollicitudin id, adipiscing quis, nulla. Sed ante orci,
sagittis at, egestas ut, egestas nec, leo. Cras malesuada dolor sed diam.
Ut lobortis velit sodales nunc.

\input table.tex

Duis vestibulum lacus eget dolor. Proin at quam. Phasellus viverra molestie
nisl. In non elit et felis commodo suscipit. Phasellus consectetuer purus.
Donec lobortis sagittis ante. Vestibulum hendrerit blandit quam.
Pellentesque habitant morbi tristique senectus et netus et malesuada fames
ac turpis egestas. Duis vel quam. Proin dictum. Donec laoreet scelerisque
nisi. Sed at urna et justo vulputate tristique. Aenean interdum. Quisque
in tortor. Morbi eget purus. Vestibulum ante ipsum primis in faucibus orci
luctus et ultrices posuere cubilia Curae; Etiam semper. 

\begin{figure}
\centerline{\includegraphics[scale=1.0]{plots/stagger.stagger.ps}}
\caption{Results.} \label{fig:results}
\end{figure}

\subsection{Results}\label{sec:results}

In Table~\ref{tab:results}, we present the results of our study.
In Figure~\ref{fig:results}, we present the results of our study.
Aliquam fermentum velit tristique turpis. Mauris mollis dapibus purus.
Quisque vulputate ullamcorper purus. Quisque aliquam rhoncus felis.
Aenean fringilla mattis mi. Ut ut massa a sem fermentum venenatis.
Fusce ut risus. Maecenas ut nisl. Praesent felis erat, tincidunt ac,
placerat vel, sodales vel, justo. Vivamus semper sollicitudin est.
Morbi id eros eu tellus dignissim viverra.

\subsection{Analysis}\label{sec:analysis}

Nulla odio. Phasellus congue malesuada quam. Pellentesque non lectus
at nisi venenatis faucibus. Proin gravida diam vitae pede. Morbi sit
amet dolor. Pellentesque feugiat tortor ut est. Ut sodales augue non
velit.

Duis arcu. Nam ac mauris. Pellentesque eleifend tellus ac eros.
Maecenas varius massa eu nunc mollis imperdiet. Donec interdum, lorem
in vestibulum feugiat, elit arcu fringilla ante, et sollicitudin ipsum
sapien a turpis. Proin sit amet nisl id metus hendrerit volutpat.

\section{Conclusion}\label{sec:conclusion}

Lorem ipsum dolor sit amet, consectetuer adipiscing elit.  Duis cursus.
Duis consectetuer, lacus id vehicula viverra, dolor quam dignissim pede,
nec ultrices metus tortor quis turpis.
Praesent non orci non est tincidunt mollis.
Nullam pellentesque auctor orci.
Morbi aliquam.
Cum sociis natoque penatibus et magnis dis parturient montes, nascetur
ridiculus mus.

In auctor ornare risus.
Nulla id magna nec pede elementum rhoncus.
Suspendisse ultricies mattis lacus.
Cras pulvinar pede nec dolor.
Lorem ipsum dolor sit amet, consectetuer adipiscing elit.
Suspendisse potenti.
Sed pede massa, scelerisque id, tempor ac, vestibulum nec, est.
Maecenas molestie tincidunt orci.
Donec ac leo quis libero iaculis semper.
Aenean ligula pede, semper eu, tristique eu, aliquam in, nisi.
Curabitur sagittis condimentum ligula.
Nunc interdum suscipit libero.


\appendices  % Indicates that appendices follow.
% If there's only one, use \appendix instead.


\chapter{Important Note About Margins}
\label{mgins}

\section{General tactics}
No two printers are alike.  Please print a few pages and \emph{measure} them.
\LaTeX\ can send commands to your printer, but it can't see whether they are
being obeyed!

Check the {\tt dvips} problem described below.
If, after tackling that,
you still need to make adjustments to get correct margins on paper
(1.5 inches on the left and 1 inch on the right),
add commands such as:
\begin{verbatim}
\hoffset = 0.5in    % shift 0.5 inch to the right
\voffset = -0.1in   % shift 0.1 inch up
\end{verbatim}
in your \verb".tex" file right after after \verb"\usepackage{gu}".


\section{The A4 problem}

If your printouts are consistently too high or too low on the page,
and you are generating PostScript files
using the {\tt dvips} command, you are probably generating output for
A4 (11.7-inch) paper.

That is the size of paper used in Europe.
American (``letter'') paper is only 11 inches long and is the size specified
for Georgetown theses and dissertations (and their electronic images).

As a quick check, try
typing the {\tt dvips} command with the argument {\tt -tletter}, like this:
\begin{flushleft}
{\tt dvips -tletter {\it filename}}
\end{flushleft}
If that cures the problem, you can fix it permanently by editing the file
{\tt config.ps} to make letter size (11-inch) paper the default.
See the {\tt dvips} documentation for details.


\chapter{Customizing the Table of Contents}

On rare occasions the table of contents will not come out formatted
the way you'd like.  Here's how to control it. \emph{Most people will
not need to do this} and can ignore these notes.

Every time you run \LaTeX, it writes the table of contents on a file
whose name ends in \verb".toc", and reads and prints the \verb".toc" file
from the \emph{previous} run.
That's why you have to run \LaTeX\ twice to get a correct table of contents.

If you need to alter your table of contents (e.g., by inserting a
page break), you can edit the \verb".toc" file in between two runs of \LaTeX.
For example, you can insert \verb"\newpage" to generate a page break.
Note that this change will not last because \LaTeX\ will write a new
\verb".toc" file ever time it runs.

Alternatively, you can add a command to your \verb".tex" file that will
add something to the contents.  For example, just before beginning
a particular chapter, you can execute the command:
\begin{verbatim}
\addtocontents{toc}{\newpage}
\end{verbatim}
to cause a page break at the corresponding place in the table of
contents.


\chapter{Common PDF Problems}

When you convert your thesis from PostScript to PDF (with
Adobe Acrobat Distiller or other suitable software), you may discover
two problems.

\section{Text shrinks when converted to PDF}

This may mean that your PostScript file was formatted for A4
(11.7-inch) paper, and PDF is shrinking it to fit on American 11-inch
paper.  See p.\ \pageref{mgins}.


\section{Type looks blurred or pixellated}

This generally means you are using the old bitmap versions of the
\TeX\ fonts rather than the PostScript versions.  As a quick check, try
adding the command \verb"\usepackage{times}" right after \verb"\documentclass".
This will change the typeface.  If the new typeface looks a lot
smoother, you've identified the problem but not cured it.

To get the PostScript \TeX\ fonts, get your system support people to
reconfigure {\tt dvips}, or download and install a newer version,
or simply run \LaTeX\ on a computer that has more up-to-date software.

Please note that PDF conversion is not affected by \verb"guthesis.sty" and the
author of \verb"guthesis.sty" cannot undertake to solve other problems with it.



\chapter{How to Cite Web Pages}

Your bibliography entries should follow whatever format is standard in your
field.
If completely stymied, see the \emph{Chicago Manual of Style} (which gives
\emph{two} formats, one for humanities and one for sciences; don't mix them
up) or the \emph{Publication Manual} of the American Psychological Association.

Even so, you will probably still be in doubt about how to cite web pages.
The way I see it, the Web is not a publisher.  Thus, a web address is not
a sufficient citation for a web page.
Instead, you must cite the document itself (indicating where and when it
originated) and then add information about the web address.

If the paper has been published, and you got a copy from the author's web
site, your job is easy.  Just do something like this:
\begin{quote}
Newton, Isaac (1672) On the refraction of coloured light.
\emph{Journal of Irreproducible Results} 23:456-789.
Web: \url{http://www.trinity.cam.ac.uk/new\-ton/colour.pdf}.
\end{quote}
That is, add the web information at the end of a normal citation.

If the paper is available only on the author's web site (at an institution
or corporation), cite the institution or corporation as the publisher,
something like this:
\begin{quote}
Newton, Isaac (1672)  On the refraction of coloured light.  Trinity College,
Cambridge.  Web: \url{http://www.trinity.cam.ac.uk/newton/colour.pdf}.
\end{quote}
The idea is to cite it just like a privately distributed paper except that
you add the web address.

If the web site \emph{is} a publisher (e.g., an online journal), then it
will probably give instructions for citation.
Do not copy numbers into a bibliography entry if you don't know what they
mean.

Consult your major professor for more advice, since the details will depend
on other aspects of your bibliography format.

Note that the character $\sim$ in web addresses is typed \verb"$\sim$",
unless you use the {\tt url} package.

Note also that by itself, \LaTeX\ will not hyphenate a URL.  It is up to you
to indicate with the symbol \verb"\-" the places where hyphens are permitted.
Always hyphenate within words, not between compound words, so that the reader
will know that the hyphen was not present in the original.

\bibliographystyle{plainnat}
\bibliography{gusample}

\typeout{***}
\typeout{*** Note!}
\typeout{*** Because this document has a table of contents,}
\typeout{*** you must run LaTeX TWICE to get it to print correctly.}
\typeout{***}

\end{document}

